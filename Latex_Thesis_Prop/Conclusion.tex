\section{Conclusion}
The current goal of development for this vowel pronunciation training tool has been to demonstrate a proof of concept for providing a useful and intuitive feedback method in a CAPT system targeting segmental errors. Based on the pedagogic theories outlined in section 1, we have concluded that L2 learners must learn to distinguish which features separate a new foreign vowel category from a similar category in their L1. Because audio playback of similar vowel may not be an effective teaching method on it's own, we assume that including a visual representation of certain features may aid L2 learners in understanding the difference between similar vowel sounds, and ultimately help them produce better examples of these vowels themselves. Visual feedback was also intended to be as straightforward as possible by showing changes in features as essentially distances and quantities. This was intended to allow the learner to quickly begin experimenting with the tool, and to free them from the burden of having to learn specific background knowledge of phonetics or phonology. We believe that this focus on simple feedback, and learner guided experimentation will allow them to identify their mistakes, and act to correct them. While there remains a great deal of work to be done in improving the design and testing its effectiveness, the initial challenge, of measuring a vowel and displaying it for feedback, has been met.
