
\section{Future Work}
The vowel training tool remains a prototype, and has not yet been used in an education setting or formally tested. In this section we wish to discuss some thoughts on future work, including an example pronunciation exercise designed with the system in mind. In addition we will cover plans for testing the system's ease of use and effectiveness in improving L2 vowel perception and production. Finally some improvements will be suggested which concerning the systems functionality, and it's general design. 

\subsection{Example Exercise}
What follows is the process envisioned for using the the vowel training tool as part of a pronunciation exercise for a non-native German speaker. Before the exercise itself begins, some initial set up would be required to tailor the task to the user's language background. The user would be asked what their native language is, and asked to read a small set of words with vowel minimal pairs. This information would be used to identify difficulty between German vowel categories and their native phonology, and to select which vowels to train on\footnote{At present the system is hard-coded to display the /\textipa{i}/ - /\textipa{I}/ distinction, but future implementations of the tool would ideally have a database of all German vowels produced by several native speakers, and could be set-up to work with any arbitrary vowel distinction.}. After the set-up, the user would be given very simple instructions informing him or her that the area to the right depicts the duration of the vowel, and the area to the left depicts a two- dimensional space showing some acoustic qualities of the vowel. They would then be told that the center of each ``bulls-eye'' is an approximate target for the acoustic quality of the German vowel depicted therein. Finally they would be told that by changing the shape of their mouth and tongue they can change where a point is plotted in the acoustic quality space, and by changing the length of the vowel, they can lengthen or shorten the bar in the duration space. After this short set of instructions, the user would be given the task of listening to some examples of minimal pairs produced by a native speaker, and then attempting to produce their own. They can use the visual feedback from the acoustic space, and the duration space to experiment with different vocal tract geometries and lengths and see their progress. After the user feels confident they can distinguish between the minimal pairs, they will be given a final post exercise pronunciation test where they will produce minimal pairs without the aid of the visual feedback. This exercise would be repeated on a regular basis and their progress tracked to monitor for improvement.

\subsection{Testing}
Testing of the tool will need to be considered from two perspectives: effectiveness and usability. To measure effectiveness, the tool must be tested for its ability to improve L2 learners pronunciation of German vowels. L2 learners would be regularly given vowel training exercises (as previously described) over an extended period of time. Recordings of their vowel productions would be saved, and their average duration and formant values would be tracked throughout the training period to see if they are approaching more typical German productions. This would indicate that the L2 learners are successfully creating vowel categories to distinguish between similar German vowels. This group would be compared to two control groups of L2 learners with a similar level of German language learning background. The first control group would receive normal classroom pronunciation training, and the second group would receive no training, so that the tool can be compared to those two baseline outcomes. The second test of the tool would be an evaluation by the L2 learners themselves on how they like the interface. This would comprise of a survey covering each part of the tool, asking how they rate ease of use and intuitiveness. An open comment section would also be provided for suggestions or changes they wish to be made. Using these two means of testing, we would hope to establish if the tool provides a benefit to L2 learners, and if they find it easy to use and understand. 


\subsection{System Improvements}
Through the development process, many areas for improvement were identified which could expand the accuracy and usability of the vowel training tool. The most pressing need is a more sophisticated system for identification of the vowel. The present algorithm limits the tools functionality to measuring only single-syllable words without the context of naturally read or spontaneous speech. For future development of the tool, we propose a forced alignment method of vowel identification which would be able to locate vowel boundaries over the duration of a phrase, rather than a single syllable. This improvement would benefit the user by training them to perceive and produce contrasts in normal speaking contexts rather than under tightly controlled conditions. In addition to more flexible vowel detection, accuracy can also be improved by tailoring the acoustic targets for vowel categories to the individual acoustic space of each user. A relatively easy first step would be to have different targets based on the user gender, as gender is known to affect average formant values \cite{patzold1997acoustic}. A more ambitious improvement would be to record reference vowels from a user at the extremes of the acoustic space, such as /\textipa{i}/, /\textipa{a}/, and /\textipa{u}/ and adjust the targets of other vowels within the acoustic space relative to those reference points. This would ensure that the users are not trying to imitate an average vowel, but one appropriate to their vocal tract physiology. These suggested improvements would require a dramatic re-design of the underlying code for the tool, which would ultimately be beyond the scope of Praat scripting. Therefore a future version of the tool would likely be designed in a more flexible language, such as Python \cite{python}. 

